\documentclass{report}

\usepackage[bithe]{mls}
\usepackage{fontspec}

\setmainfont{Fu_SM_WenJian}
\usepackage[landscape]{geometry}
\usepackage{fullpage}
\usepackage{amsmath}
\usepackage{amsfonts}
\title{bodoro arga j oyonggo be araha uheri hes'en j bithe, ujui fiyelen}
\begin{document}
\maketitle
%page 5
\chapter{mulu}%mulu notsure
\noindent 1. uju j fiyelen, oron be doktobure arga.\\
2. jai fiyelen meyen, nonggire arga.\\
3. ilan fiyelen, ekiyembure arga.\\
4. duiici meyen, kamcire arga.\\
5. sunjaci fiyelen meyen, faksalara arga.\\
6. ningguci meyen, ilan s'uwai bifi duiici s'uwai be baiire arga.\\

%page 6 I

emtelingge be ijishvn j bodoro arga, emtelingge be fudarame bodoro arga \\
kamcibuhengge be ijishvn j bodoro arga, kamcibuhengge be fudarame bodoro arga.\\
7. nadaci meyen, kamcibume ilan s'uwai arga.\\
8. jakvci meyen, ton be acabufi bihe, ton be neiihe akv j dendere arga.\\
emteli ton arga ??\\
9. uyuci meyen, holo ton be gaiifi teodejeme yargiyalara arga.\\
10. juwanci meyen, holo ton be dahvn dahvn gaiifi teotejeme yargialara arga.\\
11. juwan emuci meyen, neiicin durbejengge neiire arga. (???)\\
 12. juwan juweci meyen, ilan hos'oi arbun be bodoro uhari arga.\\
 emteli geo gu be bodoro arga.\\
 tondo hos'o akv eiiten hacin j ilan hos'oi arbun be bodoro arga.\\ 
 13. juwan ilaci arga, eiiten hacin i arbun j dere de baktalahangge be bodoro arga.\\
 eiiten hacin j tondo jecen j dere be baktakangge bodoroo arga.\\
 emteli muheren neiicin dere be bodoro arga.\\
 muheren de kamcibuhengge be bodoro arga.\\
 eiiten hacin j arbun j dere be ishunde nonggire, ekiyembure arga.\\
 encu arbun doro be ishunde duiibulere arga.\\
 14. juwan duiici meyen, tob duiibulere beye be neiicin arga.\\
 undu jecen be suwaliyame golmin durbejengge beye be neiicin arga.\\
 15. tofohoci meyen, eiiten hacin j arbun beye de baktan be bodoro arga.\\
 jergilehe derei emteli arbun j beye be bodoro arga.\\
 kamcibuhengge be bodoro arga. muhaliyan arbun de s'ongkuhe(??) be bisire emteli beyengge bodoro arga.\\
 kamcibuhengge be bodoro arga.\\
 duiibulere adali beye be ishunde nonggire ekiyembure arga.\\
 eiiten hacin beyengge kvbulibure arga.\\
 baiitalara arga.


%page 8
\chapter{ujui meyen, oron be toktobure arga}
\section{oron be toktobure arga}
oron be toktobure de, hashv erki ci ici erki baru tuwaha, 
\begin{itemize}
    \item ujui be hergen be emteli ton,
    \item jai hergen be juwei ton,
    \item ilaci hergen be tanggv j ton,
    \item duiici hergen be minggan j ton,
    \item sunjaci hergen be tumen j ton,
\end{itemize}
aiibade tuwaha ci, dulenci ujui oron be araha. ton j hergen j fejergi de emu tongki gidefi, tereci ilan oron giyalafi,
%page 9
geli emu tongki dere, oron labdu oci gemu ilata oron giyalafi, emu tongki gidalambi.\\ 
duiibuleci: <<ilan jakvn ninggun juwe nadan emu sunja ilan ninggun duiin>> sere ton j oron be toktobure de, uju de araha hergen j fejilen emu tongki gidafi, ilan oron giyalafi geli emu tongki, geli ilan oron giyalafi, geli emu tongki gida. ere uju j tongki j dergingge emteli ton, jai tongki j dergingge tumen j ton, ilaci tongki j dergingge i j ton inu.\\
ere faiidahangge << ilan tanggv jakvn ninggun i juwe minggan nadan tanggv uyunju emu tumen sunja minggan ilan tanggv ninju duiin >> sere ton inu:
\[
    386,2791,5364,
\]

\section{cyi tshun}
kamcire arga, juwe ton be ishunde kamcihe bodoro de aiikabade cyi tshun j ton oci, oron toktobure koooli julergi kooli de adali akv.\\
duiibuleci (??): emu cyi juwe tshun be, da tun j songkoi kamcire de, emu duiin, duiin sere ton bajinmbi. emu cyi juwe tshun juwe oron hergen tshun j ton. jai oron j hergen cyi i ton, ere emu cyi juwe tshun be da ton j songkoi kamcire de ujui oron j hergen jai 
%11
oron j hergen gemu tshun j ton, ilaci oron j hergen teni cyi j ton ombi.\\
jai turgun seci, juwe jijun j cyi tshun be ishunde kamcire bodoro, emu tob durbejengge dere banjinambi, banjinaha dere be emu tshun j tob durbejengge arbun emu tanggv de isinaha de teni emu cyi i tob durbejengge arbun ombi, emu cyi j tob durbejengge arbun emu tanggv de isinaha de teni emu jang ni tob durbejengge arbun ombi.\\
\begin{align}
    1\mathrm{cyi}^3 = 1000 \mathrm{tshun}\\
    1\mathrm{jang}^3 = 1000 \mathrm{cyi}
\end{align}

geli duiibuleci: ninggun jang jakvn cyi sunja tshun be duiin jang sunja cyi nadan tshun j emgi kamcime bodoci, ilan emu ilan 0 duiin sunja sere ton bahambi, baha ton j ujui oron hergen j fejilen emu tongki gidafi, dereci emte oron giyalafi, emu tongki gida, ere ujui tongki j dele arahangge tshun j ton, jai tongki j dele araha cyi j ton, ilaci tongki j dele araha jang ni ton inu. cyi j bodofi, bahangge emte jang ni tob durbejengge arbun gvsin emu, emte cyi j tob durbejengge arbun gvsin, emu tshun j tob durbejengge arbun dehi inu. nikebumre faksalarame bodorongge ton j arbun be toktoburengge julergi kooli adali.
\[
    3\underline{1}3\underline{0}4\underline{5}
\]
(0 oci << untuhun >> sere hvlanabuki?)\\
%page 12

cyi tshun i ton be da ton j songkoi kamcime bodofi, geli kamcime bodofi baha ton j oron be toktoburengge julergi de adali akv bimbi.\\
duiibuleci: ninggun cyi juwe tshun be da ton j songkoi kamcime udu ombi seci, ninggun juwe be da ton j songkoi kamcime bodofi, baha ton be geli da ton j emgi kamcime bodofi, 0 baha juwe ilan jakvn ilan juwe jakvn sere ton j uju oron j hergen j fejergi de emu tongki gidafi, ereci amasi juwete oron giyalafi, emte tongki gida, erei ujui tongki de teiisulehe hergen tshun j ton, jai tongki de teiisulehe hergen cyi j ton inu, uttu oci bahangge emu cyi j ton durbejengge beyei arbun juwe tanggv gvsin jakvn, emte tshun j tob durbejengge beyei arbun ilan tanggv orin jakvn inu. aiinu uttu seci emu tshun tob durbejengge beye arbun emu minggan de isinaha de teni emu cyi j tob durbejengge arbun banjinara, emte cyi j tob durbejengge beyei arbun emu minggan de isinaha be teni emu jang ni tob durbejengge beyei arbun banjinara turgun kai.
\[
    23\underline{8}32\underline{8}
\]

%page 15
geli duiibuleci: bele be tebure, emu tshang den ninggun cyi juwe tshun. hetu j onco duiin cyi juwe tshun, udu j onco ilan cyi duiin tshun, erei banjinara emte cy j tob durbejengge beyei arbun udu seme bodoro de, da j ton j ninggun juwe be, hetui onco j ton j duiin juwe j emke kamcime bodofi, bahara juwe ninggun 0 duiin sere ton be, geli hetui onco j ton j ilan duiin j emke kamcime bodofi, bahara jakvn jakvn sunja ilan ninggun sere ton oron be nirugan j songkoi toktobuci, emte cyi j tob durbejengge beyei arbun jakvnju jakvn emte tshun j tob durbejengge beyei arbun sunja tanggv gvsin ninggun bahambi.
\[
    3\underline{3}5\underline{6}
\]

%page 16
nikebuhe faksalaha bodoro ton j oron be doktoburengge nikebure faksalara ton bee da bihe ton j fejergi de arara de aiikabade fejergi ton, teiisulehe dergi ton ci araha oci, uthai faksalara ton i tubai hergen j teiisulehe dergi ton j oron tanggv minggan tumen ni ocibe, uthai bahara ton j ujui oron obumbi.\\
\newpage
duiibuleci:
\[
    2680.00
\]
sere da ton be emu ilan duiin sere ton j faksalaci:
\[
    \begin{array}{|lllllll}
        2&&&&&&\\\hline
        2&6&8&0&0&0 &\\
        1&3&4& & &  &\times 1000
    \end{array}
\]
bahara juwe sere ton, uthai juwe minggan inu. adarame saha seci, faksalara ton j dubei oron, da ton j minggan sere oron j fejergi de teiisulehe turgun de, tuttu bahara juwe sere ton, uthai juwe minggan inu:
\begin{align*}
    \frac{268000}{134} = 2000.
\end{align*}

geli duiibuleci: emu juwe nadan uyun nadan 0 sere da ton be, emu ilan duiin sere ton j faksalara de, 
\[
    \frac{127970}{134}
\]
fejergi j emu ilan duiin sere ton, dergi ton j emu juwe nadan sere ton ci amba ofi:
\[
    134>127
\]
faksalara ton be emu oron anaha arafi: $\frac{1279}{134}$, faksalara de bahara uyun sere ton uthai uyun tanggv inu. adarame saha seci, faksalara ton j dubei oron, da ton j tanggv j oron j fejergi de teiisulehe turgun de, tuttu bahara uyun uthai uyun tanggv inu be saha.
\[
    \begin{array}{|lllll}
        & 9 & & & \times 100\\\hline
        1&2&7&9 & \\
        &1&3&4  &
    \end{array}
\]

%page 20
\chapter{jai meyen nongire arga}
yaya ishunde nonggire ton bifi, dele emke fejilen emke arara de, urunakv gese gese salire oron be meni meni dergi fejergi de ishunde teiisulebume arambi:
\[
    +\begin{array}{|lllll}
      &  a_4 & a_3 & a_2 & a_1\\
        b_5 & b_4 & b_3 &b_2 &b_1\\\hline
    \end{array}
\]
ishunde nonggire de juwan jaluci, emke obufi julergi oron de ibebumbi .\\
<< duiibuleci:
\begin{equation}
    +\begin{array}{|lllll}
      &  a_4 & a_3 & a_2 & a_1\\
        b_5 & b_4 & b_3 &b_2 &b_1\\\hline
        &&& +1& [a_1+b_1]
    \end{array}
    \text{\setRtoL be arafi,}\; [a_1+b_1]=a_1+b_1-10, \text{\setRtoL obufi,}\; a_1+b_1 > 10\; \text{\setRtoL aika}
\end{equation}
>>\\
aiinu seci yaya emu oron de emu j ton be nonggire de juwan jaluci manggi, uthai julergi emu oron j emke ombi. uttu tumen minggan tanggv juwan ocibe, eici jang cyi tshun fun li ocibe, yaya hacin gemu ere songko.\\

duiibuleci: emu tanggv orin duiin jang uyun cyi ilan tshun de, ilan jang ninggun cyi be ishunde nonggire be, ilan be duiin j fejergi de teiisulebume arambi.\\
\[
    +\begin{array}{|rrr}
        12\underline{4} & 9 & 3\\
        \underline{3} & 6 & 0\\hline
    \end{array}
\]
%page 21
aiinu seci, duiin serengge, jang ni oron de bisire jakade, tuttu ilan jang be duiin jang ni fejergi de teiisulebume arambi. jai ilan untuhun be ishunde nonggire de ilan be kemuni ilan obume ujui oron de arahabi.
\[
    +\begin{array}{|rrr}
        124 & 9 & \underline{3}\\
        3 & 6 & \underline{0}\\\hline
         & & 3
    \end{array}
\]
tere uyun cyi ninggun cyu be ishunde nonggire de, ere ton juwan be dulere jakade,
\[
    +\begin{array}{|rrr}
        12{4} & \underline{9} & 3\\
        {3} & \underline{6} & 0\\\hline
         & 15 & 3
    \end{array}
\]
julergi oron de emke ibebumbi, juwan ci funcehe ton be, da oron j fejergi de teiisulebume, jai oron de sunja seme arahabi.
\[
    +\begin{array}{|rrr}
        12{4} & 9 & 3\\
        {3} & 6 & 0\\\hline
        \underline{+1} & 5 & 3
    \end{array}
\]
jai tere duiin jang ilan jang ni ton de neneme ibebuhe emke nonggibure jakade, tuttu ilaci oron de jakvn seme arahabi.
\[
    +\begin{array}{|rrr}
        12\underline{4} & 9 & 3\\
        \underline{3} & 6 & 0\\\hline
        8 & 5 & 3
    \end{array}
\]
da ton j juwe be kemuni juwe, emke be kemuni emke obume arahabi.
\[
    +\begin{array}{|rrr}
        \underline{12}{4} & 9 & 3\\
        {3} & 6 & 0\\\hline
        \underline{12}8 & 5 & 3
    \end{array}
\]
ere uthai nonggire de bahara emu tanggv orin jakvn jang sunja cyi ilan tshun j ton inu.\\

\newpage
geli duiibuleci: \\
gvsin duiin gin juwe yan duiin jiha juwe fun, orin gin duiin yan juwe jiha duiin fun be ishunde nonggire de,
\begin{itemize}
    \item gin j ton be gin j ton j fejilen,
    \item yan j ton be yan j ton j fejilen,
    \item jiha j ton be jiha j ton j fejilen,
    \item fun j ton be fun j ton j fejilen
\end{itemize}
arambi.
\[
    +\begin{array}{|rrrr}
        \text{\setRtoL gin} & \text{\setRtoL yan} & \text{\setRtoL jiha} & \text{\setRtoL fun}\\
        34 & 2 & 4 & 2\\
        20 & 4 & 2 & 4\\\hline
    \end{array}
\]
duiin jiha juwe jiha be ishunde nonggici, ninggun jiha be bahambi.
\[
    +\begin{array}{|rrrr}
        \text{\setRtoL gin} & \text{\setRtoL yan} & \text{\setRtoL jiha} & \text{\setRtoL fun}\\
        34 & 2 & \underline{4} & {2}\\
        20 & 4 & \underline{2} & {4}\\\hline
        & & 6 & 6
    \end{array}
\]
juwe yan duiin yan be ishunde nonggici, ninggun yan be bahambi.
\[
    +\begin{array}{|rrrr}
        \text{\setRtoL gin} & \text{\setRtoL yan} & \text{\setRtoL jiha} & \text{\setRtoL fun}\\
        34 & \underline{2} & 4 & {2}\\
        20 & \underline{4} & 2 & {4}\\\hline
        & 6 & 6 & 6
    \end{array}
\]
aiikabade fun j ton jaluci juwan oci, emu jiha ibebu. jiha j ton jaluci juwan oci emu jiha ibebu. jang j ton juwan ninggun ome jaluci, emu gin obu. jai duiin 0 untuhun be ishunde nonggici, duiin be bahambi.
\[
    +\begin{array}{|rrrr}
        \text{\setRtoL gin} & \text{\setRtoL yan} & \text{\setRtoL jiha} & \text{\setRtoL fun}\\
        3\underline{4} & {2} & 4 & {2}\\
        2\underline{0} & {4} & 2 & {4}\\\hline
        4 & 6 & 6 & 6
    \end{array}
\]
ilan juwe be ishunde nonggici, sunja be bahambi.
\[
    +\begin{array}{|rrrr}
        \text{\setRtoL gin} & \text{\setRtoL yan} & \text{\setRtoL jiha} & \text{\setRtoL fun}\\
        \underline{3}{4} & {2} & 4 & {2}\\
        \underline{2}{0} & {4} & 2 & {4}\\\hline
        54 & 6 & 6 & 6
    \end{array}
\]
gemu nirugan de araha songkoi ara. uheri ton susai duiin gin ninggun yan ninggun jiha ninggun fun ombi.
%page 22
\end{document}